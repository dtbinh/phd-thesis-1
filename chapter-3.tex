\setcounter{chapter}{2}

\chapter{Models of lexicon formation}
\label{c:thesis-overview}

In this thesis we will systematically analyze the performance of
different classes of lexicon formation models. Starting simple, we
will confront our agents with more and more challenging communicative
tasks and each time examine what additional representational
mechanisms and learning strategies are required to reach communicative
success and coherence. In doing so, we will follow the increasing
complexity of models that we laid out in Section
\ref{s:nature-of-form-meaning-couplings} (page
\pageref{s:nature-of-form-meaning-couplings}).

In order to be able to evaluate the impact of the additional
challenges stemming from embodiment and real-word perception, we will
first investigate lexicon formation in simulated worlds: throughout
Part \ref{p:II}, agents will be presented with idealized simulated
perceptions of varying complexity. To set the stage, we will briefly
review models for the naming of individual objects in Chapter
\ref{c:naming-game}.  From there on, we discuss strategies for dealing
with ambiguity that arises from conceptualization, multi-word
utterances and structured meanings in Chapter \ref{c:gg}. Motivated by
these results, we introduce a flexible model of word meaning
representation and learning in Chapter \ref{c:flexible}. 

Part \ref{p:III} then discusses embodied models of lexicon
formation. We will first introduce the robotic experimental setup with
its mechanisms for visual perception and social cognition in Chapter
\ref{c:embodiment}. Then we look at how robots can construct notions
of object individuality as a prerequisite for aligning sets of proper
names in Chapter \ref{c:gng}, and how more general categories can be
grounded trough language games in Chapter
\ref{c:grounded-categories}. Finally, in Chapter
\ref{c:flexible-grounded} we apply the model from Chapter
\ref{c:flexible} to real-word embodiment and
analyze its performance.\\



\section{Guessing the meaning of novel words}

Because there is no direct relationship between word forms and
referents and due to the nature of words as arbitrary relationships
between meanings and forms, hearers are faced with the challenge of
guessing the meaning of novel words. When the hearer does not know a
word (and can not infer its meaning using the other words in the
utterance and the context), then he non-linguistically signals a
communicative failure and the speaker will then point to the intended
referent. Although this pointing will unambiguously establish shared
reference, the hearer does not know which aspect of the referent is
covered by the unknown word: it could be its color, its size, or even
a combination thereof.

The degree of this uncertainty depends on the nature of the
coupling between meaning and form. When single word forms map to
single categories that stand for unique referents as a whole, then
there is no uncertainty for the hearer at all and he just needs to
associate a novel word to his conceptualization of the individual
object. But as soon as words refer to categories such as \texttt{red}
or \texttt{small}, hearers need to infer which category was meant upon
hearing a novel word. This problem gets multiplied when the language
game involves multi-word utterances (and when thus it is not clear
which word covers which part of the meaning) or when word meanings are
allowed to be structured (a word can refer to single categories or
combinations of categories, see Section
\ref{s:nature-of-form-meaning-couplings} on page
\pageref{s:nature-of-form-meaning-couplings}).


The challenge of dealing with this uncertainty is usually linked to
the problem of \emph{referential indeterminacy} and a thought
experiment carried out by \cite{quine60word}: he discussed an imagined
situation in which a field linguist tries to learn a language
unfamiliar to him from a native speaker. As they walk through a
forest, they encounter a rabbit and the native points to it and says
the word ``gavagai''. The linguist then forms the reasonable
hypothesis that the word means \texttt{rabbit}, but Quine makes point
that he can not be sure what the meaning of ``gavagai'' is and that
there is potentially an infinite number of possible meanings: it could
mean ``Let's go hunting'', ``There will be a storm tonight'', ``dinner'',
and so on.


Children also face the problem of referential uncertainty when
learning their mother tongue. Nevertheless, they learn words
extraordinarily quickly, from only a very few or even one
exposures. This phenomenon is called ``fast-mapping'' and was
extensively studied by \cite{carey78child}. Although it can take years
for children to home in to the proper meaning of words in all their
nuances, children make very good initial guesses about what words
refer to. In the literature, there is an enormous amount of empirical
studies showing that children prefer some interpretation of novel
words over others. For example \cite*{akhtar96role} showed that in a
object naming task with toy objects, 24-month-old children tend to
associate unknown words with objects that are novel in the
context. Similarly, \cite*{smith96naming} demonstrated that
three-year-old children rely on relative saliency when selecting
features for learning names for objects with attached parts.


Citing these findings, many researchers have concluded that children
thus must be endowed with (possibly innate) word learning
\emph{biases} or \emph{constraints} \citep{markman92constraints,
  gleitman90structural}. In this theory, constraints greatly reduce
the hypothesis space of possible meanings and only due to that make
the task of learning a language achievable. For example
\cite{macnamara82names} proposed the \emph{whole object bias}:
children assume that a novel label is likely to refer to the whole
object and not to its parts, substance, or other
properties. Furthermore, \cite*{landau98object} suggested the
\emph{shape bias} -- children initially use object shape as the main
categorization ground and only later on incorporate other properties
such as its function. And with the \emph{mutual exclusivity
  constraint} \citep{markman88childrens}, children assume category
terms are mutually exclusive, i.e. a novel word can not refer to a
property of an object for which the child already knows a
word. Similarly, \cite{clark87principle} formulated the
\emph{principle of contrast} (every two forms contrast in meanings)
and the \emph{principle of conventionality} (for each meaning, there
is a conventional form that speakers expect to be used in the language
community).\\


\noindent All these studies clearly show that children indeed consistently
prefer some interpretations of novel words over others and as we will
see, implicitly using some of these strategies such as the principle
of contrast or the mutual exclusivity constraint in our lexicon
formation experiments will also help to reach coherence in the
population. There is, however, a debate whether it is necessary to
assume language specific biases or constraints to explain these
empirical results or whether they can be the consequence of other,
possibly more general, cognitive mechanisms.


Most prominently, \cite{tomasello03constructing,tomasello99cultural}
argues that no special mechanisms are needed and that word learning to
a large extend relies on the children's general ability to understand
the intentions of their caregivers in naturally occurring social
interactions \citep{tomasello01perceiving} and in the motivation to
participate in joint activities and to share psychological states with
others \citep*{tomasello05understanding}. We share the stance that
``These findings are consistent with the view that fast mapping
emerges from a general capacity to learn socially transmitted
information -- including, but not limited to, the meanings of words''
\citep[p. 34ff]{bloom00how-children}.


Others have explained children's word learning skills with the ability
to observe statistics in co-occurrences between objects and words, a
theory called \emph{cross-situational learning}. For example
\cite{akhtar99early} presented children with novel objects that varied
in shape and texture. By consistently labelling objects of similar
properties ``a modi one'', children associated the quality that
remained constant across trials to the new word. In a more recent
study, \citep{smith07infants} showed similar effects for associating
novel words to more holistic concepts such as \texttt{ball} and
\texttt{dog}.


Inspired by this empirical evidence, scholars such as
\cite{siskind96cross-situational} and \cite*{smith06cross-situational}
operationalized their understanding of cross-situational learning in
computational studies on lexicon formation. In this technique, a
learner initially derives a set of possible \emph{candidate} meanings
from the context and stores all of them with a novel word. In
subsequent exposures to the same word in other contexts, the hearer
eliminates all those meanings that are not consistent with the context
(i.e in the intersection with the meanings derived from the current
context) until unambiguous mappings are found. There are, however, a
number of problems with this approach. Requiring observation of many
word - context pairs, the time to gain usable word meanings by far
exceeds the number of exposures that children need on average. Second,
in order to enumerate all possible meanings of a novel word and for
using the technique of intersecting meanings across contexts, the
learners need fully established category systems that do not change,
which is often not the case when for example robotic agents co-evolve
their ontologies with lexicons in the learning process. Consequently,
most computational experiments on cross-situational learning have been
done in simulated worlds where the environment already provides
pre-conceptualized atomic meanings that are shared between speaker and
hearer (with exceptions such as in
\citealp*{debeule06cross-situational}). Third and finally, models of
cross-situational learning usually consider single-word utterances and
unstructured word meanings. Scaling to more complex communicative
challenges as introduced in this thesis has proved to be difficult
\citep{vogt03investigating}.\\

\noindent In general, we find the notion of a \emph{hypothesis} space that gets
\emph{pruned} over the course of many interactions problematic. We
will show in this thesis that lexicon formation models that consider
word learning as an enumeration and subsequent elimination of
alternative hypotheses will not scale well with increasing population
sizes, meaning spaces, and the challenges of embodiment. Instead we
will argue for models in which learners construct and gradually shape
word meanings \citep{bowerman01shaping}. The hypothesis is that
``\dots the use of words in repeated discourse interactions in which
different perspectives are explicitly contrasted and shared, provide
the raw material out of which the children of all cultures construct
the flexible and multi-perspectival -- perhaps even dialogical --
cognitive representations that give human cognition much of its
awesome and unique power'' \citep[p. 163]{tomasello99cultural}.
Although in this view learners also make guesses at the meaning of
novel words, they are different in nature. Children cannot have at
hand all the concepts and perspectives that are embodied in the words
of the language they are learning -- they have to construct them over
time through language use. ``For example, many young children
overextend words such as \emph{dog} to cover all four-legged furry
animals. One way they home in on the adult extension of this word by
hearing many four-legged furry animals called by other names such as
\emph{horse} and \emph{cow}'' \citep[pp
73--74]{tomasello03constructing}.



\section{Scaling, robustness and the challenge of real-world
  perception}


The second focus of this thesis is on \emph{scaling} and
\emph{robustness} of lexicon formation models. We will investigate how
well models perform with increasing communicative challenges and by
that try to find the boundaries in which they are applicable. Many of
the models reviewed here only have been tried in ``easy enough''
environments and tasks and we will systematically analyze under which
conditions they fail and why. Most importantly, we will test all our
models with respect to how well they scale with larger population
sizes. Virtually every model in the literature works properly when the
number of agents in the population is two, because then each agent is
part of every interaction and conventions thus become easily
shared. But whereas many models scale well to small population sizes
of 5 to 10 agents, they often become impractical for populations of
100 or 1000 agents due to fundamental shortcomings in the way how
words and meanings are represented and processed. Similarly, scaling
with meaning spaces is an issue. We will evaluate population dynamics
in worlds with increasing number of objects and rising complexity and
structuredness of (simulated) perceptions.


On the other hand, we will argue that under the condition that some
crucial dynamics are in place, lexicon formation models are robust
with respect to the particular strategies chosen for invention,
adoption and alignment. For example, many algorithms have been
proposed for updating the inventories of agents after an interaction
based on the outcome of the game, and we will dissect which of them
are really required to reach success and coherence. \\


\noindent Related to scaling and robustness is the issue of the
influence of real-world perception on lexicon formation models. First
of all, not providing agents with shared symbolic perceptions adds the
additional complexity of category formation, which creates new kinds
of dynamics when ontologies and lexicons are constructed in parallel
and interdependently. For example, strategies for updating word
confidences also need to take into account that underlying categories
also may have shifted their meaning. Or when an interaction fails,
there is the hard decision to make whether the categories involved
were inappropriate or whether the word forms were simply not
conventionalized.
 
Furthermore, embodiment creates other kinds of uncertainties that need
to be dealt with. Agents can view a scene from different angles,
lighting conditions may vary and thus the perceptions that two
different robots have of the same physical object will never be the
same. Even a single robot will perceive an object differently over the
course of time due to camera noise, robot motion and general
uncertainty in computer vision systems. Nevertheless, human concepts,
such as, for example, the color red, are robust to such influences --
we will recognize an object as red under very different lighting
conditions and even subjects with color deficiencies are often able to
communicate about colors.

Concretely, embodied lexicon formation models need to cope with
\emph{perceptual deviation}, i.e. that specific continuous features
(e.g. position, shape, width and height, color information, etc.)
computed by the vision system for an object differ drastically between
the perception of speaker and hearer. For example one robot might
perceive the height of an object as being {\tt 0.72} and the other one
as {\tt 0.56}. This will inevitably cause each agent to have a
different notion of a word such as ``high''. We will make the point
that investigating lexicon formation with real robots leads to more
robust and realistic models, which in turn also perform better in
simulated environments.







%%% Local Variables: 
%%% mode: latex
%%% TeX-master: "phdbook"
%%% End: 
