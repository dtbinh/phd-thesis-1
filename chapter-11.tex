
\setcounter{chapter}{10}

\chapter{Summary \& discussion}
\label{c:chapter-11}

In this thesis we systematically analyzed the performance of different
classes of lexicon formation models both in an simulated environment
and with an embodiment in physical humanoid robots. Starting from
simple lexicon representations and strategies for language processing,
learning and alignment, we confronted our agents with increasingly
more challenging communicative tasks and examined each time what
additional representational mechanisms and learning strategies were
required in order to reach communicative success and coherence.

We evaluated each of the models with respect to their alignment
dynamics and scaling and especially focused on how well the models can
cope with the challenges of referential uncertainty and real-word
perception. To allow for a just comparison between models, we tried
(where possible) to keep most of the parameters and machinery constant
across all of the experiments. Throughout this thesis, we used the
same population size, the same language game script, the same
diagnostics and repair strategies for learning from failed
interactions, the same performance measures, and the same world
simulation respectively robotic interaction scenario.



\section{Word meaning representations and referential uncertainty}

We started our investigations in Part II with a series of experiments
in which all aspects related to language grounding such as perception,
categorization and social interaction were scaffolded. Instead, agents
were interacting in a simulated environment in which objects
consisted of pre-conceptualized (sets of) symbols such as
\texttt{object-1}, or \texttt{category-2}. As a consequence, the
meanings to be expressed were already ``in the world'' and thus were
also immediately shared by all agents of the population.

~\\

\inparagraph{Individual objects, unstructured meanings, single-word
  utterances} To set the stage and to introduce the building blocks of
all language game experiments in this thesis, in Chapter
\ref{c:naming-game} we briefly introduced the \emph{Naming Game},
which is the most simple and most studied lexicon formation model in
the literature. The task in this model is to learn to associate single
word forms to atomic, unstructured meanings which are provided by a
shared simulated environment. 

Words are represented as scored mappings between forms and
meanings. Because different agents can independently invent different
word forms for the same objects, agents will adopt many of these forms
and consequently there is a competition between words that share their
forms (\emph{synonymy}) and alignment mechanisms are needed that
eliminate competing words. \emph{Lateral inhibition} can be used to
achieve this, and we showed that (at least for a wide range of values)
the choice for the actual score update parameters is not at all
crucial for reaching coherence as long as there is some feedback loop
that makes sure that a successful use in communication increases the
likelihood of a word to be used in future interactions.

Referential uncertainty does not play a role because when an agent
does not know a word form, then pointing will reveal the object, which
itself then um-ambiguously serves as the meaning to be associated to
the novel form. Consequently, the model scales very well with
increasing population size and scales even linearly with the number of
objects in the world.

~\\

\inparagraph{Categories, unstructured meanings, single-word
  utterances} Next, in Section \ref{s:sgg-sw-unstructured} we took
away the scaffold that the wold simulator provides agents with unique
object identifiers. Instead, perceived objects are characterized by
sets of categories and therefore \emph{conceptualization} mechanisms
that find sets of categories that discriminate the topic from the
other objects in the context need to be added. Production and parsing
strategies now select from multiple possible conceptualizations of a
scene and we compared several strategies with respect to the number of
meanings that are used in production and adoption. We showed that none
of them has clear advantages, but that it is beneficial when learning
and alignment mechanisms take alternative paths in the semiotic
network into account.

In this first category-based experiment we limited word meanings to
single categories and allowed agents to use only one word per
utterance. With our chosen word simulation parameters (15 categories,
10 categories per object, between 2 and 5 objects per scene), this
means that conceptualization fails in about 45\% of the cases because
no single category can be found that is solely applicable to the
topic.

Besides that, referential uncertainty arises to an extend that hearers
hearing a novel form must pick from potentially many candidate
categories and due to that, lexicons start containing multiple
associations from the same form to different meanings (sometimes
called \emph{homonymy}), which are then later dampened using lateral
inhibition. This slight increase in complexity already caused agents
to take significantly longer to reach communicative success and
coherence, agents enumerated a lot of different mappings in their
lexicons before they got pruned by alignment, and already this simple
model does not scale well with increasing number of categories in the
world.


~\\

\inparagraph{Categories, structured meanings, single-word utterances}
In the next experiment in Section \ref{s:sgg-sw-structured} we allowed
word forms to be associated to sets of categories while still keeping
the limitation of one word per utterance. Using structured word
meanings resulted in 100\% discriminative success because speakers
could now use multiple categories to distinguish a topic from the
other objects in the context.

Nevertheless, because with single-word utterances each different
combination of categories needs to be expressed by a different word
forms, agents accumulate hundreds of different words in their lexicons
without reaching any communicative success or coherence, which
illustrates that single-word utterances for structured meanings is
obviously a bad strategy.

~\\


\inparagraph{Categories, unstructured meanings, multi-word utterances}
As an intermediate step towards the full complexity of the final
experiment of Chapter \ref{c:gg}, in Section
\ref{s:sgg-mw-unstructured} we first analyzed a model with multi-word
utterances for unstructured meanings. That is, the same word
representations as in Section \ref{s:sgg-sw-unstructured} are used,
but production and parsing mechanisms were enabled to deal with
multi-word utterances.

One additional challenge lies in the recovering from partial
processing, that is when a speaker only knows words for some parts of
the utterance or when a hearer only knows meanings for some of the
words in the utterance. But more importantly, multi-word utterances
introduce \emph{interdependent alignment dynamics}: how well a
convention spreads in the population does not only depend on how well
it was used in previous interactions, but also on the other words that
it was used with together in the utterance. When an interaction fails,
the agents can not know which word or words of the utterance was
responsible for the communicative failure and consequently it can
happen that well-conventionalized words become punished as part of the
alignment process. Nevertheless, the gain from being able to
discriminate in 100 percent of the interactions balances these
difficulties so that the overall alignment and scaling dynamics are
similar compared to the model with unstructured word meanings and
single-word utterances.


~\\


\inparagraph{Categories, structured meanings, multi-word utterances}
The full-blown complexity of ``Guessing Game''-like experiments was
then investigated in Section \ref{s:sgg-mw-structured}, in which
multi-word utterances were combined with word meanings consisting of
sets of categories. In addition to the ambiguity of deciding which
words cover which categories, referential uncertainty drastically
increases because now there is also ambiguity in \emph{specificity}:
Upon hearing a novel word, agents need to decide whether the word
refers to a single category, a combination of categories, or the
complete meaning as a whole.

As a result of that, agents enumerate large numbers of words with high
degrees of synonymy and homonymy in their lexicons and that need to be
eliminated by lateral inhibition in order to reach
coherence. Alignment is additionally made difficult by the fact that
the feedback loop from communicative success to the lexicon becomes
less reliable: In the first 2000 interactions, agents communicate
successfully although different word meanings were used in up to 8
percent of the cases, causing the wrong word associations to be
increased and decreased by lateral inhibition.

Although with the default world simulation parameters and the standard
population size of 10 agents the overall alignment dynamics look
similar to the previous experiments (with delayed success and higher
intermediate lexicon sizes), performance quickly breaks down when
moving beyond the boundaries of these parameters. We showed that for
increasing population sizes and for increasing context sizes (which
increases both the number of alternative conceptualizations and
therefore referential uncertainty as well as the average number of
categories and therefore the ambiguity in which parts of meanings
words cover), success in the game and coherence is reached not at all
or only after thousands of interactions in which agents do not
communicate successfully at all. The high variance in performance
across different experimental runs indicates that agents go through
extended periods of \emph{random search} until some words start being
successfully used by a critical fraction of the population.

Furthermore, the lateral inhibition dynamics constitute a bias towards
unstructured word meanings. Although agents have the capability to
present and process structured word meanings and although indeed many
words are initially connected to sets of categories, only words that
cover single categories survive in the lexicons of the agents, which
we attributed to the disadvantage of interdependent alignment dynamics
that words with structured word meanings have.

~\\


\inparagraph{Categories, flexible word representations} Finally, in
Chapter \ref{c:flexible} we introduced a lexicon formation model that
addresses these shortcomings by capturing uncertainty in the
representation of word meanings themselves. Instead of having
competing mappings to different sets of categories for the same word,
words now have flexible connections to different categories that are
constantly shaped by language use, which we achieved by keeping an
\emph{(un)certainty score} for every category in a form-meaning
association instead of scoring the meanings as a whole. By allowing
the certainty scores to change, the representation becomes adaptive
and the need to explicitly enumerate competing hypotheses disappears.

We showed that agents which use such representations and the
accompanying processing and alignment strategies enjoyed high
communicative success from early on, with conservatively growing
lexicons that contain stable structured word meanings. And repeating
the scaling experiments from the previous chapter, we demonstrated
that the model easily copes with the same increasing communicative
challenges that the lateral inhibition base models struggled with. 

We identified two key factors for this drastic increase in
performance. First, the similarity based lexicon application allows
agents to communicate successfully even when word meanings are not yet
conventionalized. Both speakers and hearers are able to ``stretch''
their existing word meanings to uses that are far away from the actual
meanings of the words, because for the successful interpretation of an
utterance it enough that the overall similarity of the words in the
utterance to the topic is higher than to the other objects in the
context. And second, the similarity-based alignment mechanisms allow
agents to gradually refine and shift the meanings of their words to
better conform future uses, without having to eliminate competing
hypotheses on word meanings.



\section{Grounded word meanings and challenges from real-word
  perception}

In the third part of this thesis we applied the lexicon formation
models from the previous three chapters to real-world situated
interactions of two Sony humanoid robots. The key difference to the
experiments in simulated environment is that the world does not
provide pre-conceptualized categories anymore, and instead
distinctions such as small vs. big, red vs blue, thing vs. toy. vs
teddy bear have to be constructed from the raw sensory experiences of
the robots, which adds further complexities to the mechanisms for
constructing and maintaining semiotic networks and which introduces
new challenges such as perceptual deviation.

Programming robots to play language games about objects in their
environment is in itself a very difficult but also very interesting
engineering task and in Chapter \ref{c:embodiment} we documented our
various state-of-the-art solutions to problems of visual perception,
object tracking, joint attention and social interaction, their
integration into a robust whole system and the overall experimental
setup that allowed us to conduct repeated and controlled embodied
language games experiments.

~\\

\inparagraph{Grounded object identity, single word utterances} In
Chapter \ref{c:gng} we then investigated what it takes to extend the
Naming Game from Chapter \ref{c:naming-game} to our robotic setup. We
endowed our agents with the ability to capture the invariant
properties of sensory experiences of objects with prototypes, that is
points in the sensory space that are applied using a nearest neighbor
computation and that adapt and shift in order to better capture the
statistical distributions of visual object features across multiple
perceptions of the same object.

However, individual physical objects can drastically change their
appearance, both over time and within the same scene when viewed by
the two robots from different angles, and as a result agents end up
establishing multiple prototypes for different ``views'' of the same
physical object. In order to successfully construct mental
representations of individual objects (and therefore real proper
names), additional heuristics are needed. We demonstrated that by
exploiting temporal-spatial continuity and the lexicon itself as
sources for associating separate prototypical views of the same
physical object to an individual, the number of words in the lexicons
of the agents matched the actual number of distinct physical objects
in the world.


The non-grounded Naming Game is the simplest lexicon formation model
that can be imagined and therefore proved to be an ``E. coli
paradigm'' for investigating alignment strategies, mathematical proofs
of convergence, impact of network structure and so on. Furthermore, it
has also led to views that proper names are semantically simpler than
words for kinds of objects (e.g. ``red'' or ``block'') and that they
might be precursors of compositional communication systems (as for
example in \citealp{steels05emergence}). Nevertheless, we showed that
the dynamics of the Grounded Naming Game differ drastically from the
non-grounded version and that the underlying semantics of proper names
are much more complex, which suggests that proper names are ``more
likely to be late developments in the evolution of language. In the
historical evolution of individual languages, proper names are
frequently, and perhaps always, derived from definite descriptions, as
is still obvious from many, such as, \emph{Baker}, \emph{Wheeler},
\emph{Newcastle}'' \citep[p. 266]{hurford03neural}.



~\\

\inparagraph{Grounded categories, competing form-meaning mappings}
Next, in Chapter \ref{c:ggg} we tried to apply the lateral inhibition
based lexicon formation models from Chapter \ref{c:gg} to our embodied
setup, which required two changes to the mechanisms for constructing
and maintaining of semiotic networks: First, agents need to be able to
construct ontologies of meaningful \emph{perceptual categories} such
as \texttt{red} and \texttt{small} from their sensory experiences. And
second, word alignment dynamics need to take into account that each
agent individually constructs such categories from noisy perceptions
and thus the success of words in the population also depends on how
conventionalized the underlying categories are. In order to
demonstrate that the learning and alignment mechanisms are independent
from the chosen categorization strategy, we implemented two different
category representations (one based on discrimination trees and a
second using prototypes on single sensory channels) and showed that
the interplay of categories and words indeed works well when
interlocutors artificially have the same perception of a scene.


However, when the scaffold of shared perception is removed, alignment
dynamics more or less break down. Agents continuously adopt new word
meanings and thus do not reach stable lexicons and high communicative
success (maximum 60\% with discrimination trees and 70\% with
prototypes), which we explained with the high degree of perceptual
deviation. Differences in the visual perception of physical objects by
the two interacting agents frequently prevent interlocutors from
successfully applying already conventionalized word meanings, which in
turn increases the problem of ``wrong'' feedback from communicative
success (interactions often fail although very similar categories were
used and many interactions succeed with very different underlying
meanings).


A lexicon formation model that tries to select from alternative form
meaning couplings is therefore not applicable to a scenario where
inconsistent categorization happens as a result of perceptual
deviation. This also explains the low overall communicative success
for a population of 5 agents in the Perspective Reversal experiment
\citep*{steels09perspective-alignment,loetzsch08typological}, which
had similar word alignment strategies and an embodiment in Sony Aibo
robots. But more importantly, although also very similar (and even
improved) word representations and learning mechanisms were used, this
means that we were not able to reproduce the results of the Talking
Heads experiment \citep{steels99situated} using our robotic setup. We
speculate that this is due to the fact that perceptual deviation did
not play a role in that experiment because the two robotic cameras
were looking from almost the same angle at objects on a whiteboard and
thus speaker and hearers always had a very similar perception of a
scene.


~\\

\inparagraph{Grounded categories, flexible word representations} The
flexible word representations and alignment strategies from Chapter
\ref{c:flexible} are the only lexicon formation model throughout this
thesis that could be applied to embodied agents without
modification. As we showed in the (therefore short) Chapter
\ref{c:gfwm}, the overall performance and scaling behavior looks
almost identical to the non-grounded version despite the same levels
of perceptual deviation as in the previous experiments. High
communicative success is reached within relatively few interactions,
and rich word meanings of different specificity emerge that reflect
the distribution of object properties in the world. This is possible
because the model is not only able to capture the uncertainty of what
words mean but also the uncertainty of how to categorize objects
across different contexts in the word meaning representation itself.


~\\

\noindent Many, if not even most experiments in the field of
artificial language evolution that follow a constructivist cultural
learning approach (including most of the experiments on the emergence
of grammatical communication systems or the grounding of richer
semantics) use some kind of lateral inhibition-based alignment
strategies and therefore often fail to scale beyond very simple
communicative tasks. Given our results, we argue that language
learning should not be considered as an enumeration and subsequent
elimination of alternative hypotheses but rather as a process in which
learners construct and gradually shape their conceptual and linguistic
inventories over time. New members of a linguistic community that try
to learn the language do not spend years of randomly searching a
hypothesis space before they start being able to communicate
successfully. Instead, we acquire simple operational approximations of
novel linguistic items very quickly and refine them later on over the
course of repeated interactions.




\section{Acknowledgements}

The research reported here was carried out at the Sony
Computer Science Laboratories in Paris and Tokyo and at the Artificial
Intelligence Laboratory Vrije Universiteit Brussel and it was funded
by the European FP6 project \emph{ECAgents} and the FP7 project
\emph{ALEAR}. Sincere thanks go to Masahiro Fujita, Hideki Shimamura,
and their team at Sony CSL Tokyo for allowing us to work with the
fantastic Sony humanoid robots and their support.

~\\

\noindent The work presented in this thesis could not have been
carried out without the joint effort of a great team, and I especially
want to thank three people that who had a substantial part in
designing and setting up the experiments.

First, I did not enter a new completely field with this thesis but
could already rely on the extensive body of research by Luc Steels and
his previous collaborators. Next to teaching me how to program in Lisp
and of course to organizing funding, in 2007 together we began a
project of ``deconstructing'' language games that formed a starting
point for the experiments in Chapter \ref{c:gg}. Furthermore, he was
deeply involved in the design of the Grounded Naming Game (Chapter
\ref{c:gng}, \citealp*{steels12grounded-naming-game}).


Second, there was a very fruitful and inspiring collaboration with
Pieter Wellens from the VUB AI lab. Next to many other joint work, we
developed the machinery for conducting language game experiments that
is underlying all experiments in this thesis (as part of the Babel2
framework; \citealp*{loetzsch08babel2,steels10babel}), and we devised
the flexible word meaning representations and applied them to our the
robotic setup (Chapters \ref{c:flexible-grounded} and
\ref{c:flexible},
\citealp*{wellens08flexible,wellens12multi-dimensional}).


And third, all work on the Sony humanoid robots from Chapter
\ref{c:embodiment} \citep*{spranger12perceptual} was done together
with Michael Spranger from the Sony CSL Paris. He had a major share in
developing the visual object tracking system and together we built all
the machinery for having the robots play language games, conducted
endless series of interactions, and analyzed the resulting data to
create the data sets used in this thesis.


Moreover, I want to thank all the other colleagues at Sony CSL and the
VUB AI Lab for great discussions and advice, but more importantly, for
being a fantastic group to work with (in order of appearance):
Frederic Kaplan, Verena Hafner, Pierre-Yves Oudeyer, Joachim De Beule,
Wouter Van den Broeck, Joris Bleys, Vanessa Micelli, Simon Pauw,
Katrien Beuls, and Katja Gerasymova.


Finally, I have to thank the Mathematisch-Naturwissenschaftliche
Fakultät II of the Humboldt-Universität zu Berlin for setting a strict
time limit for submitting PhD-theses, which eventually forced me to
hand in mine.













%%% Local Variables: 
%%% mode: latex
%%% TeX-master: "phd-thesis"
%%% End: 
